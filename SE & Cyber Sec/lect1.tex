\documentclass[10pt,a4paper]{article}
\usepackage[utf8]{inputenc}
\usepackage[english]{babel}
\usepackage{amsmath}
\usepackage{amsfonts}
\usepackage{amssymb}
\usepackage{graphicx}
\usepackage{inputenc}
\author{Gérard Tio Nogueras}
\title{SE & Cyber Security}
\begin{document}
"Hopefully you'll be able to catch up"
\section{Intro}
Software eng security can theoreticly be achieved by following the pillars:
\begin{itemize}
\item risk mgnt
\item touchpoints
\item knowledge
\end{itemize}
\subsection{Touchpoints}
There are 7 main touchpoints that follow the software creation:
\begin{itemize}
\item Code review
\item Arch risk analysis
\item Pentesting
\item Risk-based sec testing
\item abuse (misuse)
\item Security requirements
\item Security operations
\end{itemize}

SDL (sec dev lifecycle) has the following steps:
\begin{itemize}
\item Training \\
You need to keep updated with all the news because every 4 years most of the information is obsolete.
\begin{itemize}
\item Core Security Training
Security knowledge is lacking in general\\
Historically not taught in universities\\
Barely taught in universities now\\
The attackers are getting smarter every day, security is a
moving target\\
Annual training is necessary\\
Need to cover all the topics in the SDL\\

\end{itemize}
\item Requirements (organisation requirements)
\begin{itemize}
\item Establish Security Requirements
Assign a security champion.\\
Assign a privacy advisor (external to team).\\
Assign a privacy lead (internal to team).\\

\item Create quality Gates/Bugs bars
Define and document a bug bar for sec and privacy.\\ Classification of what are moderate, important , or critical sec and privacy bug types.\\
User to set priority for fixing and to determine if the product can ship\\
Ensure that bug reporting tools can track security  and privacy	issues and that a database can be  queried	dynamically for all security bugs at any  time
\item Security and privacy risk assessment\\
Start a security plan. Identify timing and resources for  SDL steps, such as:
\begin{itemize}
\item Team training
\item Threat modeling
\item Security push
\item Final security review
\end{itemize}
Security Risk Assessment:
\begin{itemize}
\item Identify specific functional areas that  need special review
\end{itemize}
Privacy Risk Assessment:
\begin{itemize}
\item Stores personally identifiable information (PII) on the  user's computer or transfers it from the user’s computer  (P1)
\item Provides an experience that targets children or is  attractive to children (P1)
\item Continuously monitors the user (P1)
\item Installs new software or changes file type associations,  home page, or search page (P1)
\item Transfers anonymous data (P2)
\item None of the above (P3)
\end{itemize}

\end{itemize}
\item Design\\
This step works as a cycle:
\begin{itemize}
\item Design Requirements
\item Threat modeling
\item Attack surface Analysis/reduction
\end{itemize}

\subsubsection{Design Requirements}
\paragraph{Design Principles}
\begin{itemize}
\item Least privilege
\item Compartmentalization
\item Validate/Sanitize external input to the system 
\item Log/audit system and data access
\item Reuse sec components and libs
\item secure the weakest link
\end{itemize}
\paragraph{Specifications}
\begin{itemize}
\item Secure architecture
\item Identification of sec critical components
\item Secure functional requirements
\item Security failures
\end{itemize}
\paragraph{Attack surface Analysis/reduction}
It is impossible to stop all the attacks since depending on the time and money invested anything is possible. The idea is to reduce the attackers surface.\\
This surface has 3 dimensions:
\begin{itemize}
\item target / enablers
\item channels and protocols
\item access rights
\end{itemize}
One of the tools used to deal with this is ASA(attack surf analyzer)
\end{itemize}
\section{Security Usability}
\subsection{Why security fails ?}
\begin{itemize}
\item User doesn't consider security as a main task
\item The background of users 
\item Bad knowledge of the system by the users and therefore don't associate the risks.(unique password, and anything lowering the sec layer)
\item Professionals can also be blamed if the design of the system doesn't allow users to easily implement the secure requirements.
\end{itemize}
\subsection{Importance}
You must be really aware of the weakest link of your systems, usually the humans are the weakest link, because they are potential victims of phishing, ....\\
The design must be easy but secure.
\subsection{Guidelines}
The guidelines for best usability are the following ones:\\
\begin{itemize}
\item easy to learn (easy use every time the users has to interact, if too hard users won't comeback.)
\item efficiency to use
\item easy to remember 
\item less error prone (able to adapt if users do stuff unexpected)
\item likeable 
\end{itemize}
\subsection{Security Usability}
Usability qualities(usual solution are passwords, that is why there is so much emphasis on this subject) + security = happy people
3 papers to read.
\section{Passwords}
Can people remember 56 bit passwords ? (12 English letters)
\subsection{How to store data in human brain ?}
We can see our brain as an hard drive, the difference is that we forget things after a long period of time if we don't recall it regularly.\\
According to famous cryptographers, humans aren't build to remember proper keys.\\
\subsection{So how to remember passwords}
We have to go through spaced repetition:\\
Paste it many times in your brain and then enter a period of copy/paste period.\\
One technique used is adding a verification number that would be delayed every time the user would fail the password. There is an interesting paper to read to train the user's memory for their master password (for a password manager)
\end{document}