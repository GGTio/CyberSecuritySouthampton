%%%%%%%%%%%%%%%%%%%%%%%%%%%%%%%%%%%%%%%%%
% Simple Sectioned Essay Template
% LaTeX Template
%
% This template has been downloaded from:
% http://www.latextemplates.com
%
% Note:
% The \lipsum[#] commands throughout this template generate dummy text
% to fill the template out. These commands should all be removed when 
% writing essay content.
%
%%%%%%%%%%%%%%%%%%%%%%%%%%%%%%%%%%%%%%%%%

%----------------------------------------------------------------------------------------
%	PACKAGES AND OTHER DOCUMENT CONFIGURATIONS
%----------------------------------------------------------------------------------------

\documentclass[12pt]{article} % Default font size is 12pt, it can be changed here

\usepackage{geometry} % Required to change the page size to A4
\geometry{a4paper} % Set the page size to be A4 as opposed to the default US Letter

\usepackage{graphicx} % Required for including pictures

\usepackage{float} % Allows putting an [H] in \begin{figure} to specify the exact location of the figure
\usepackage{wrapfig} % Allows in-line images such as the example fish picture

\usepackage{lipsum} % Used for inserting dummy 'Lorem ipsum' text into the template

\linespread{1.2} % Line spacing

%\setlength\parindent{0pt} % Uncomment to remove all indentation from paragraphs

\graphicspath{{Pictures/}} % Specifies the directory where pictures are stored

\begin{document}

%----------------------------------------------------------------------------------------
%	TITLE PAGE
%----------------------------------------------------------------------------------------

\begin{titlepage}

\newcommand{\HRule}{\rule{\linewidth}{0.5mm}} % Defines a new command for the horizontal lines, change thickness here

\center % Center everything on the page

\textsc{\LARGE University of Southampton}\\[1.5cm] % Name of your university/college
\textsc{\Large Msc Cyber Security}\\[0.5cm] % Major heading such as course name
\textsc{\large Foundation of Cyber Security}\\[0.5cm] % Minor heading such as course title

\HRule \\[0.4cm]
{ \huge \bfseries Privacy Laws}\\[0.4cm] % Title of your document
\HRule \\[1.5cm]

\begin{minipage}{0.4\textwidth}
\begin{flushleft} \large
\emph{Author:}\\
Gerard \textsc{Tio Nogueras} % Your name
\end{flushleft}
\end{minipage}
~
\begin{minipage}{0.4\textwidth}
\begin{flushright} \large
\emph{Supervisor:} \\
Pr. Vladimiro \textsc{Sassone} % Supervisor's Name
\end{flushright}
\end{minipage}\\[4cm]

{\large \today}\\[3cm] % Date, change the \today to a set date if you want to be precise

%\includegraphics{Logo}\\[1cm] % Include a department/university logo - this will require the graphicx package

\vfill % Fill the rest of the page with whitespace

\end{titlepage}

%----------------------------------------------------------------------------------------
%	TABLE OF CONTENTS
%----------------------------------------------------------------------------------------

%\tableofcontents % Include a table of contents

%\newpage % Begins the essay on a new page instead of on the same page as the table of contents 

%----------------------------------------------------------------------------------------
%	INTRODUCTION
%----------------------------------------------------------------------------------------

\section{Introduction} % Major section
So we have two questions to discuss, the first one\\\\Simply put, it is not possible to de-identify individual personal data unless you are talking about a single, low resolution measurement in total isolation. For example, you might be able to de-identify a single blood pressure reading like 138/75 and add it to an aggregate with a low-resolution timestamp but it is not possible to de-identify more than that in our hyper-connected world that never forgets anything.\\\\
\textbf{Critically discuss the above statement by assessing what the implications of such a statement would be for law-makers wanting to better delineate the domain of data protection laws.}\\\\
And the second\\\\
WP29: A specific  pitfall  is  to consider pseudonymised  data  to  be  equivalent to anonymised  data.\\ICO:  Pseudonymisation is a way to produce anonymised data but on an individual-level basis\\\\
\textbf{Are these statements inconsistent? Critically discuss with references to both documents. }



\section{Content}
\subsection{1st Part}

%------------------------------------------------

\subsubsection{Understanding the statements} % Sub-section

Analyse each part of the statement for better comprehension, then group them all together for a global comprehension of the statement. Afterwards explain the example and create one to assure understanding of the statement.

%------------------------------------------------

\subsubsection{Implications of these statements} % section

Explain thoroughly how this statement is important for privacy laws, who is it going to affect, what exactly are the implications. A lot of referencing to the papers + try to find current open questions/cases to add understanding.  

%------------------------------------------------

\subsubsection{Discussion of these implications} % Sub-sub-section

Discuss and argue about the subject, not too personal, use logic and papers information here.

\subsubsection{Personal}

Bring personal thoughts enforced by papers, give ideas as solutions, everything not brought up earlier can come here.


%------------------------------------------------

\subsection{2nd Part} % Major section

%------------------------------------------------

\subsubsection{Understanding the statements}

State who the WP are and why they don't believe pseudo =/= anonymous.

State who the UK ICO is and their arguments on pseudo == anonymous at an individual-level basis.

What do they mean by individual-level basis, give example

\subsubsection{Implications of these statements}
explain in depth both citations using papers arguments and using example and own words

\subsubsection{Discussion of these implications}

\subsubsection{Personal}



%----------------------------------------------------------------------------------------
%	CONCLUSION
%----------------------------------------------------------------------------------------

\section{Conclusion} % Major section

Here sum up all the discussion, and give final opinion.

Example citation \cite{Figueredo:2009dg}.

%----------------------------------------------------------------------------------------
%	BIBLIOGRAPHY
%----------------------------------------------------------------------------------------

\begin{thebibliography}{99} % Bibliography - this is intentionally simple in this template

\bibitem[Figueredo and Wolf, 2009]{Figueredo:2009dg}
Figueredo, A.~J. and Wolf, P. S.~A. (2009).
\newblock Assortative pairing and life history strategy - a cross-cultural
  study.
\newblock {\em Human Nature}, 20:317--330.
 
\end{thebibliography}

%----------------------------------------------------------------------------------------

\end{document}