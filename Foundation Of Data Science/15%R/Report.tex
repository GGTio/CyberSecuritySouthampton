%%%%%%%%%%%%%%%%%%%%%%%%%%%%%%%%%%%%%%%%%
% Simple Sectioned Essay Template
% LaTeX Template
%
% This template has been downloaded from:
% http://www.latextemplates.com
%
% Note:
% The \lipsum[#] commands throughout this template generate dummy text
% to fill the template out. These commands should all be removed when 
% writing essay content.
%
%%%%%%%%%%%%%%%%%%%%%%%%%%%%%%%%%%%%%%%%%

%----------------------------------------------------------------------------------------
%	PACKAGES AND OTHER DOCUMENT CONFIGURATIONS
%----------------------------------------------------------------------------------------

\documentclass[12pt,twocolumn]{article} % Default font size is 12pt, it can be changed here

\usepackage{geometry} % Required to change the page size to A4
\geometry{a4paper} % Set the page size to be A4 as opposed to the default US Letter

\usepackage{graphicx} % Required for including pictures

\usepackage{float} % Allows putting an [H] in \begin{figure} to specify the exact location of the figure
\usepackage{wrapfig} % Allows in-line images such as the example fish picture

\usepackage{lipsum} % Used for inserting dummy 'Lorem ipsum' text into the template

\linespread{1.2} % Line spacing

%\setlength\parindent{0pt} % Uncomment to remove all indentation from paragraphs

\graphicspath{{Pictures/}} % Specifies the directory where pictures are stored

\begin{document}

%----------------------------------------------------------------------------------------
%	TITLE PAGE
%----------------------------------------------------------------------------------------

\begin{titlepage}

\newcommand{\HRule}{\rule{\linewidth}{0.5mm}} % Defines a new command for the horizontal lines, change thickness here

\center % Center everything on the page

\textsc{\LARGE University of Southampton}\\[1.5cm] % Name of your university/college
\textsc{\Large Msc Cyber Security}\\[0.5cm] % Major heading such as course name
\textsc{\large Foundation of Data Science}\\[0.5cm] % Minor heading such as course title

\HRule \\[0.4cm]
{ \huge \bfseries Statistics with R}\\[0.4cm] % Title of your document
\HRule \\[1.5cm]

\begin{minipage}{0.4\textwidth}
\begin{flushleft} \large
\emph{Author:}\\
Gerard \textsc{Tio Nogueras} % Your name
\end{flushleft}
\end{minipage}
~
\begin{minipage}{0.4\textwidth}
\begin{flushright} \large
\emph{Supervisors:} \\
Pr. Elena  \textsc{Simperl}\\ % Supervisor's Name
Dr. Chris \textsc{Phethean}\\
Dr. Ramine \textsc{Tinati}\\
Dr. Markus \textsc{Brede}
\end{flushright}
\end{minipage}\\[4cm]

{\large \today}\\[3cm] % Date, change the \today to a set date if you want to be precise

%\includegraphics{Logo}\\[1cm] % Include a department/university logo - this will require the graphicx package

\vfill % Fill the rest of the page with whitespace
\end{titlepage}
\newpage
\section{First step}
We start by plotting the distribution of the times of catch and the distribution of the weights of the fishes caught.
Let us look at the information we can retrieve from these distributions following the lectures recommendations. One of them  is to find the good bin sizes using the Freedman-Diaconis rule which gives us 2.8h for the times and 0.61kg for the weights.
\subsection{Typical Scores}
We start with the mean, the median, the mode and finally the geometrical mean.
\textbf{Times}
\begin{itemize}
\item mean 9.388
\item median 8.950
\item mode 1.65
\item geometrical mean 6.786465 
\end{itemize}
\textbf{Weights}
\begin{itemize}
\item mean 1.8431
\item median 1.8250
\item mode 3.22
\item geometrical mean 1.36948
\end{itemize}
\subsection{Range of Scores}
Here we will look at the range of the distributions, their spread and how they spread.
\textbf{Times}
\begin{itemize}
\item minimum 0.010
\item maximum 23.160
\item variance 31.99831
\item standard deviation 5.656705
\item interquartile range 8.335
\item skewness 0.2477339
\begin{enumerate}
\item old
\item new
\end{enumerate}
\item kurtosis -0.8870124
\end{itemize}
\textbf{Weights}
\begin{itemize}
\item minimum 0.0100
\item maximum 4.2300
\item variance 1.162248
\item standard deviation 1.078076
\item interquartile range 1.8
\item skewness 0.08966
\begin{enumerate}
\item old
\item new
\end{enumerate}
\item kurtosis -1.169334
\end{itemize}
\subsection{Kernel density estimation}
Here we plot a continuous estimation of the PDF (probability density function) for our variables. This allows for a smoother estimate of the distribution then the histogram. Normally this is used to extrapolate the PDF for a larger population.
Strangely for the weights distribution the sum of the KDE values exceeds 1 by far which seems a bit odd since it is supposed to estimate a PDF. On the other hand the "times" KDE seems perfectly normal
\section{Second step}

\end{document}